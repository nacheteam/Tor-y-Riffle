Una vez ya están todos los repetidores necesarios configurados y activos en la red Tor se pueden empezar a obtener circuitos para cada uno de los usuarios. Los circuitos son un conjunto de repetidores entre los cuales se incluye uno de entrada y otro de salida, el resto de los mismos serán nodos relays o middle nodes.\\
En la configuración del circuito se busca entre los nodos aquellos que tienen como flag Guard indicando que son susceptibles de ser escogidos como nodos de entrada. Lo mismo se hace para escoger el nodo de salida. Entre todas las posibilidades se escoge aquel que otorga una latencia y ancho de banda más parecido a la conexión del usuario de Tor. El resto de nodos son escogidos de manera más laxa para acabar de configurar el circuito de conexión del usuario. Para esta elección entre nodos el usuario antes de decidir se descarga toda la información pública de los nodos de la red y es el propio ordenador del usuario el que decide el circuito a usar.\\
El circuito tiene que tener como es lógico una longitud mínima de 3 nodos para poder contener un nodo de cada tipo, aún así la longitud del circuito no tiene por qué ser 3 sino que es de longitud variable añadiendo así mas capas de redireccionamiento del tráfico.\\
Si una persona controlara muchos nodos de tipo exit y de tipo guard se podría hacer una correlación con el paso del tiempo entre las personas que navegan por la red Tor y las peticiones que estos requieren de los servidores. Es por esto que los circuitos tienen una caducidad y se van rotando todos los nodos guard y nodos exit. Así mismo todos los nodos pueden ser de cualquier tipo en un principio. Es por esto que los circuitos son difíciles de controlar para un atacante. Si tienen un tiempo de vida limitado aunque algún atacante controle el circuito en 5-10 minutos este cambiará por completo por lo que seguirle la pista a un usuario de la red Tor se hace complejo. Si hiciéramos nodos dentro de la red Tor con la intención de abarcar una posición concreta en un circuito como nodo de salida o nodo de entrada no lo conseguiríamos porque en primer lugar los nodos que ocupan posiciones tan críticas son analizados cuidadosamente y en segundo lugar los nodos van rotando de posición sin informar a la persona que alberga el repetidor.\\
La comunicación es iniciada en el circuito por el usuario que desea navegar por internet y este añade las capas de cifrado necesarias en función del número de nodos que tenga el circuito. Este circuito mantiene una conexión durante el tiempo que se necesite o hasta que se alcance la caducidad del circuito. Cuando el paquete llega al nodo de salida este hace la petición y como la conexión sigue establecida el paquete que el servidor devuelve puede recorrer el mismo camino a la inversa. Los nodos dentro del circuito sólo pueden saber de donde llegan los paquetes y a dónde lo deben enviar. Debido a esto el usuario que comienza la navegación nunca es accesible desde el punto de vista del servidor al que se mandan peticiones.