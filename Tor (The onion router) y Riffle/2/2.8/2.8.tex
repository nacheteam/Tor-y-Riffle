Tor, aún siendo una buena propuesta de seguridad y anonimato, no es una red segura por completo. Hay ciertas vulnerabilidades de seguridad importantes que sufre la red Tor, ya sea de manera real o teórica. A continuación vamos a presentar algunos ataques y vulnerabilidades que sufre la red Tor.

\subsubsection{Correlación de punto a punto}
Tor ha montado una red completa para el uso desde el exterior, es decir, el uso desde el internet general. La seguridad de Tor recae sobre la propia red y no sobre sus fronteras, por lo que en los extremos de la red Tor nosotros podemos monitorizar el tráfico.\\
Supongamos que somos capaces de controlar nodos de la red Tor y tuviéramos en nuestro poder un número importante de nodos tanto de entrada como de salida. La idea que surje es clara, el nodo de entrada conoce la identidad de la persona y el nodo de salida conoce la información de la comunicación. Aquí es donde podemos intentar establecer una correlación entre la información de entrada y la información de salida. Por ejemplo a través de identidades con las que nos conectemos a servicios por medio de la red Tor o por el contenido en general de los paquetes de red enviados y filtrados en el nodo de salida.\\
Esta técnica se dice que ya ha sido utilizada por agencias como la NSA para el control de la pedofilia, tráfico de armas, tráfico de droga y trata de personas a través de la red Tor. Aunque esta vulnerabilidad existe no es un ataque inexpugnable. La protección ante este ataque podría ser, por ejemplo, el uso de un servicio oculto de la red Tor. Con este servicio el tráfico nunca sale de la red y por tanto la comunicación permanece segura.

\subsubsection{Pérdida de información del nodo de salida}
Como ya hemos comentado anteriormente una de las partes más críticas de la red Tor son los nodos de salida. En concreto este ataque se basa en una de las cosas dichas anteriormente en las características de los nodos de salida. La información en los nodos de salida no va encriptada por lo que, si tenemos un nodo Tor alojado en nuestro ordenador y este nodo consigue el flag de salida exit podemos analizar todo lo que enviamos a la red.\\
El potencial problema de este ataque recae en que se pueden obtener credenciales de inicio de sesión de los servicios a los que el cliente esté intentando acceder. Este ataque fue descubierto por Dan Egerstad que fue capaz de  obtener los credenciales de cuentas de correo electrónico de usuarios de la red Tor empleando este método.

\subsubsection{Bloqueo de nodos de salida}
Este problema no es en realidad una vulnerabilidad de la red Tor en cuanto a seguridad sino un problema de diseño de la misma. El proyecto se ha preocupado por las personas que quieran acceder a la red Tor y tengan los nodos de entrada bloqueados o filtrados (en su empresa, país o por su ISP) mediante los puentes. Este problema plantea justo la misma situación pero con los nodos de salida. Por ejemplo, páginas de ciertos estados o bancos no son accesibles desde la red Tor porque las IPs de los nodos de salida son bloqueadas por dichas empresas. Este problema hace por tanto que en ciertas circunstancias o en ciertas webs la red Tor no sea utilizable.

\subsubsection{Filtración de identidad a través de BitTorrent}
En este caso tenemos dos perspectivas del problema: el "Bad Apple Attack" y la exposición de IPs.\\
En la primera perspectiva algunos investigadores se dieron cuenta de que utilizando BitTorrent dentro de la red Tor con una aplicación no segura y no preparada para el uso de Tor se podía descubrir la IP del usuario de la aplicación de BitTorrent. Esta aplicación intentará conectarse al ordenador "Bad Apple" para descargar un fichero o compartirlo y cederá su IP al atacante. Con este ataque quedó expuesto casi el 40$\%$ del tráfico de la red Tor que precisamente consiste en compartir ficheros mediante torrent.\\
En la segunda perspectiva se va aún más lejos permitiendo, gracias a la recopilación de IPs con el método anterior, hacer un secuestro de sesión o un Man in the Middle pudiendo así atacar al usuario de torrent. Esto como ya he mencionado antes hace que la transmisión de ficheros por la red Tor no sea totalmente segura pudiendo saltarse con ello las protecciones y filtros de anonimato de Tor.

\subsubsection{Ataque DDOS a Tor}
El ataque de denegación de servicio es algo a lo que ni siquiera Tor escapa. Este ataque, como ya es conocido, intenta mediante peticiones masivas interrumpir el funcionamiento de un servidor o en este caso de un nodo de la red. Al ser la información de la red Tor pública se pueden obtener las IPs de los nodos de manera simple y por tanto, si se tuvieran las suficientes máquinas para hacer dicho ataque, se podría cortar el servicio de muchos o todos los nodos de la red haciendo por tanto que no funcione. Este ataque se podría particularizar sobre los nodos de entrada o salida y con ello ya quedaría  inutilizada. El mecanismo de protección de Tor ante esto es el dinamismo de los nodos de salida y entrada (cambian con frecuencia) y el número tan amplio de nodos a los que se debería de atacar para interrumpir el servicio.

\subsubsection{HeartBleed}
Esta conocida vulnerabilidad de SSL sobre HTTPS afectó también a la red Tor que estuvo varios días sin ofrecer servicio mientras renovaba todas sus claves y todos sus certificados con la intención de invalidar la posible información que los atacantes hubieran obtenido. Este ataque sigue siendo aplicable si hay aún alguna web HTTPS que no tenga el parche de seguridad que arregló la vulnerabilidad.