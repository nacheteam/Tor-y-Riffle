Como ya comentamos al principio la red Tor tiene como objetivo principal el hecho de anonimizar a  la persona cuando navega por internet y permitir el acceso seguro a la red incluso en países que filtren algunas webs o todas. A colación de esta pregunta surge un problema de diseño de la red. Si los nodos publican toda su información incluyendo su ip, entonces simplemente con bloquear una gran parte de nodos de la red Tor o la mayoría de los nodos de entrada o salida ya estaremos filtrando el acceso a la red de los usuarios. Este problema a día de hoy ya está resuelto y a dicha solución se la conoce como puentes o bridges en inglés.\\
Un puente dentro de la red Tor es un nodo de acceso a la red que no publica su información y cuya ip va variando cada cierto tiempo. Esto hace que en un principio estos nodos no se puedan bloquear porque no se conoce su información, y aún sabiendo la ip esta cambiará cada cierto tiempo haciendo que cualquier lista negra se quede obsoleta.\\
Esta idea de oscurecer los nodos es un buen comienzo, pero, ¿cómo se conocen por tanto las ips de los nodos para poder configurar nuestra conexión a Tor? La solución dada de momento es consultar una base de datos que proporciona una ip por usuario que solicite un puente. Si la página web que soporta dicha base de datos también está bloqueada en el país del usuario, entonces existe un correo electrónico que interpreta comandos enviados mediante e-mail y devuelve las ips de los puentes solicitados.\\
Gracias a esta solución actualmente países como China, Afganistán o Siria pueden utilizar la red Tor para comunicarse con el exterior.