Los descriptores son documentos que almacenan información de importancia acerca de los nodos en la red Tor. Estos documentos ayudan a la elección de nodos en la red, a saber si es un servicio oculto o cuál es la configuración de dicho nodo. Estos documentos son públicos y como tal pueden ser leídos por cualquiera. Los archivos correspondientes a los nodos son albergados por las Autoridades de Directorio que controlen la parte de la red Tor a la que pertenezca dicho nodo.\\
Los tipos de descriptores que hay dentro de la red son:\\
\begin{itemize}
	\item Server Descriptor: este fichero contiene la información principal de los repetidores. Entre otras cosas se pueden obtener de los mismos sus políticas de salida, ancho de banda, dirección IP, su puerto de conexión OR, sistema operativo,etc. Este fichero es el que un usuario de la red descarga a la hora de decidir los nodos que van a participar en su circuito.
	\item ExtraInfo Descriptor: en este fichero ya se ofrece la información completa del nodo así como sus ficheros de configuración. Este archivo no es descargado por los usuarios por el aumento considerable de tamaño. Aún así, si queremos que nuestro cliente de Tor reciba la información completa de los nodos, podemos configurarlo para ello y en vez de descargarse el fichero Server Descriptor accederá al ExtraInfo Descriptor.
	\item Micro Descriptor: si en vez de utilizar el fichero de Server Descriptor que puede resultar lento en versiones actualizadas del navegador el cliente ya no descarga el archivo Server Descriptor sino Micro Descriptor. Este es básicamente una versión simplificada del original con la intención de que se intercambie menos información al inicio de la comunicación del usuario con la red Tor.
	\item Network Status Document: las Autoridades de Directorio son las encargadas de generar estos ficheros que contienen una valoración de un nodo hecha por las Autoridades. Este fichero es el llamado fichero de consenso que es crucial para el buen funcionamiento de la red ya que no sólo califica la velocidad y configuración sino las posibles intenciones maliciosas del nodo. Este fichero es actualizado con periodicidad para comprobar la integridad de la red a lo largo del tiempo. Los Network Status Documents están compuestos a su vez de ficheros Router Status Entry.
	\item Router Status Entry: estos ficheros incluyen la información sobre cada uno de los repetidores de la red. La información de estos ficheros es proporcionada por cada una de las Autoridades de Directorio de la red. En estos ficheros podemos encontrar los flags que se le van a asignar a los nodos y los cálculos heurísticos de selección de nodos a la hora de formar circuitos.
	\item Hidden Service Descriptor: es el documento que identifica a un servicio oculto. Este fichero es firmado y publicado por el mismo servicio oculto para garantizar la seguridad de la información. Estos documentos son facilitados a los nodos con el flag HSDir que se encargarán de ser asignados como puntos introductorios. Dentro del fichero hay información acerca de la comunicación con el servidor (obtención del punto rendezvous), configuración del servicio oculto, etc.
\end{itemize}