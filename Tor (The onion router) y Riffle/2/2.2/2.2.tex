Cuando un nodo Tor es iniciado por primera vez pasa por un largo proceso de evaluación y testeo por parte de la red. La configuración de un nodo depende de un fichero llamado torrc el cual contiene directivas que señalan cómo se tiene que comportar el nodo, cuales son sus métodos de seguridad, ancho de banda, ...\\
Las etapas van medidas en días y son las siguientes:
\begin{itemize}
	\item Fase uno(Día 0 a día 3): en esta primera etapa las autoridades de directorio hacen que no participemos en muchos circuitos dentro de la red. Esto es así porque en un principio las Autoridades no saben cuales son nuestras intenciones y quieren probarnos primero. Los circuitos normales que nos proporciona la red Tor para participar se intercalan con circuitos de prueba que las propias Autoridades generan. A través de estos circuitos se mandan pulsos de vez en cuando que intentan abarcar el máximo de la banda ancha de nuestro internet para poder tener una medición real de cual es nuestra velocidad. Después de 45 minutos de conexión continuada se nos comienzan a asignar flags con la intención de comenzar a calificar el nodo.\\
	\item Fase dos(Día 3 a día 8): en esta fase ya está nuestro nodo funcionando como un middle relay de manera normal y están empezando los testing para comprobar si seríamos un buen nodo guard. Hay que tener en cuenta que aunque estas fases estén marcadas por días puede que un nodo nunca pase de esta segunda fase o incluso de la primera por problemas de configuración o por la propia intención del que alberga el nodo de no ser nunca un nodo guard o nodo exit. Es también en esta fase donde comienzan las comprobaciones por si el nodo está configurado para ser nodo de salida.\\
	\item Fase tres(Día 8 a día 68): durante esta fase nuestro nodo ya puede ejercer como nodo middle y como nodo guard. Se nos recomendarán métodos de seguridad para nuestro nodo y se afianzará en la red. Esta fase no significa que nuestro nodo sea un nodo guard por completo aún, ya que todavía se nos está sometiendo a testing para esta tarea. Estos testing incluyen circuitos de prueba para medir seguridad y velocidad en la red, ya que si recordamos del apartado anterior necesitaremos como mínimo 250KB/s en ambos sentidos(subida de datos y descarga) de banda ancha. Todas estas comprobaciones también se hacen en el caso de que el nodo sea de salida y esté configurado para ello.\\
	\item Fase cuatro(Día 68 en adelante): dentro de esta fase nuestro nodo es un nodo guard por completo y hemos pasado todas las fases de comprobación. En esta fase aún así una vez cada cierto tiempo(pueden ser días o semanas) nuestro nodo pasará unas ciertas pruebas para comprobar que el nodo no es malicioso y continúa teniendo la misma configuración y banda ancha. Si nuestro nodo era de salida en este paso ya estará funcionando como un nodo de salida genuino y recibirá muchos mas tests y pruebas que el resto de nodos por ser este el de mayor importancia dentro de la red. Se harán pruebas como por ejemplo hacer peticiones aleatorias de páginas web y ver si este nodo inyecta información dentro de los paquetes ya que es ese punto del circuito el único lugar ajeno al usuario en el cual los paquetes van sin cifrar.
\end{itemize}