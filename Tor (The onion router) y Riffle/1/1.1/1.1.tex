Tor es una red anónima gestionada por el equipo de desarrollo Tor Project. Dicha red implementa en así mismo un protocolo de comunicación entre sus componentes que luego analizaremos en profundidad.\\
La intención de Tor es que los usuarios de internet puedan tener una navegación completamente anónima. Para ello se sirve tanto de cifrados en varias capas como de una red interna parcialmente descentralizada que se encarga de gestionar y enrutar el tráfico de red. El hecho de que sea parcialmente descentralizada viene dado porque parte de la red son nodos configurados por particulares mientras que el resto son configurados por la propia organización.\\
El protocolo está basado en TCP. Esto como ya veremos más adelante es porque Tor mantiene una línea de comunicación abierta dentro de su red por un tiempo determinado para cada usuario, es decir, crea un camino en el grafo de su red interna para cada usuario.\\
Tor puede funcionar desde la consola del sistema y gestionar ahí su funcionamiento pero lo más común es utilizar el navegador proporcionado por el Tor Project Team. Este navegador se encarga de toda la configuración de Tor y los bridges necesarios. Como curiosidad el navegador está basado en Mozilla Firefox.\\
Comúnmente en la red se confunde a Tor con la Deep Web. Durante el desarrollo de esta tecnología han ido surgiendo diferentes páginas web dentro de Tor con funcionalidades más o menos legales, pero hemos de recordar que el germen de Tor fue la navegación anónima en Internet y no su uso para actividades ilegales.