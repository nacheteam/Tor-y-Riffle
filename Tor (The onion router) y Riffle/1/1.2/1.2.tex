En términos de anonimato en la red ya existían previamente los proxys. De manera rápida un proxy es un nodo en internet a través del cual redireccionamos nuestro tráfico de red en ambos sentidos. Esto nos permite ocultar en cierto modo a la persona con la que nos comunicamos quienes somos o la posición geográfica de nuestro ordenador. Esto es porque la comunicación entre servidores de páginas web y el usuario o cualquier otro servicio de internet se hace directamente con el nodo u ordenador que implementa el proxy, es decir, el ordenador del usuario del proxy no aparece involucrado directamente en la comunicación con los servicios.\\
Tor se puede ver como un proxy masivo y dinámico en la red, aún siendo mucho mas que eso. Tor utiliza muchos nodos en internet que se redireccionan entre sí y tienen la misma filosofía que el proxy, ocultar los datos de quien realmente está haciendo la conexión. A parte de eso en cada redireccionamiento entre nodos tenemos una capa de cifrado diferente lo que añade un nivel más de seguridad. Así mismo como se podrían monitorizar los nodos Tor en internet y así romper al cabo de un rato analizando las comunicaciones la  privacidad de sus usuarios, los redireccionamientos cambian para prevenir esto.\\
Por lo tanto Tor tiene una idea similar en cierto modo a la de un proxy pero con muchos mas niveles de seguridad que le añaden una gran ventaja con respecto a los mismos y que indagaremos en ello más adelante.