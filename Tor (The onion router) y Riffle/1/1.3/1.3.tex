SOCKS es un protocolo que se encarga de que la transmisión de paquetes entre un cliente y un servidor sea a través de un servidor proxy. Este protocolo es un estándar de facto.\\
Su relación con Tor es evidente. SOCKS hace que una conexión mantenida(basada en TCP) sea segura apoyándose en un proxy. Si replicamos esta idea de proxys y encriptamos la comunicación tendremos algo parecido a lo que es Tor.\\
Las diferencias entre SOCKS y Tor son:
\begin{itemize}
	\item SOCKS se basa en el uso de un único proxy para la comunicación mientras que Tor emplea un número variable de ellos.
	\item Tor utiliza SOCKS en intervalos de tiempo, es decir, SOCKS tiene una conexión persistente mientras que Tor limita dicho tiempo de conexión y va variando los proxys empleados.
	\item SOCKS delega la encriptación y comunicación segura al servidor proxy el cual puede o no tener una buena configuración de seguridad. Por el contrario Tor se encarga de cifrar la comunicación y de que cada nodo en un circuito no conozca al resto.
\end{itemize}