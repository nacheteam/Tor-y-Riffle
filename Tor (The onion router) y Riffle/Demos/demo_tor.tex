La demostración que hemos preparado para Tor ha consistido en montar un nodo tor y mantenerlo en ejecución durante una semana siendo monitorizado con la aplicación de consola ARM.\\
ARM es un monitor de nodos Tor que da información muy valiosa acerca del mismo. Es capaz de decirte los recursos que está consumiendo dicho nodo en la máquina y cuales son las gráficas de uso de la red. Con ello podemos observar que cantidad de información estamos siendo capaces de rediredcionar a través de nuestra red. También nos dice, aunque de manera menos precisa por motivos de diseño de la red Tor, los circuitos en los que estamos participando como nodo así como la configuración de nuestro nodo. Así mismo uno de los campos más importantes y con lo que hemos podido probar y ver nuestro desarrollo teórico de Tor ha sido mediante los flags.\\
Para la configuración del nodo Tor se deben de establecer los parámetros que se deseen en un fichero llamado torrc. Este fichero contiene las directivas de configuración de un nodo Tor cualquiera, entre las cuales están por ejemplo qué puertos usar o cuanta banda ancha ofrecemos para el nodo. En nuestro caso, teníamos que tener precauciones al configurar nuestro nodo de prueba. En la red Tor aunque no se quiera se mueve mucho contenido altamente ilegal en España como pornografía infantil, contrabando de drogas y armas y comunicaciones de bandas ilegales. Si nuestro nodo hubiera sido un nodo de salida, nuestro ordenador habría sido el encargado de hacer las peticiones y recolectar información de las páginas web con contenidos ilegales para facilitarlos al usuario de Tor, por lo que en términos legales nosotros habríamos sido los que han infligido un acto delictivo. Es por eso que nuestro nodo lo configuramos con la política "reject *:*" la cual indica que se rechaza cualquier puerto de cualquier IP para las funciones de nodo de salida. El resto de configuraciones que hicimos no fueron demasiado complejas ya que la documentación de los archivos torrc está muy bien explicada y su uso es bastante sencillo.\\
Durante nuestro experimento el nodo adquirió los flags V2Dir, Running y Valid. Estos flags como ya explicamos indican que nuestro nodo era compatible con la segunda versión del protocolo de Autoridades de Directorio, que el nodo estuvo en ejecución estable por al menos 45 minutos y que no estábamos ejecutando una versión maliciosa de Tor ni estábamos en ninguna lista negra. Estos flags fueron el fruto de una conexión de unos 100KB/s de subida y 100KB/s de bajada mantenida, como ya hemos dicho, durante 7 días.\\
El monitor ARM es muy expresivo a la hora de comunicarte las comprobaciones que se realizan sobre tu nodo acerca de la velocidad, accesibilidad y adquisición de flags por lo que pudimos observar como nuestro nodo era calificado por las Autoridades. También nos fijamos que cualquier nodo posee un nombre establecido por el usuario que no es identificativo del mismo y un fingerprint que es el usado dentro de la tabla hash interna de la red Tor para manejar los nodos.\\
ARM como ya hemos comentado también da información de los nodos que conforman el circuito aunque dicha información no es precisa. Aún así nos permitió ver cómo son las configuraciones de un nodo Tor de salida o entrada y la variedad de países que participan en el proyecto Tor instalando nodos.\\
Así mismo hemos recurrido a páginas como "Tor status" para ver la actividad de nuestro nodo y qué posición ocupábamos dentro del ranking de nodos. Esta misma página nos permitía comprobar los datos de tráfico de red que nos estaba dando ARM con los que la página nos proporcionaba, viendo así que el nodo funcionaba de manera correcta. Por último para hacer un análisis más general de tor hemos acudido al portal de "Tor metrics" del propio Tor Project. En este portal podemos comprobar datos como cuantos nodos hay activos, cuales son los temas que se encuentran en Tor o los países que más emplean la red entre otras cosas. Como curiosidad también encontramos la web "Tor flow" que proporciona un gráfico mundial del movimiento de información dentro de la red.